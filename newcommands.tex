% scalable bullet (need graphicx package)
\newcommand\sbullet[1][.5]{\mathbin{\vcenter{\hbox{\scalebox{#1}{$\bullet$}}}}}

% widebar (need amsmath package)
% https://tex.stackexchange.com/questions/16337/
%   can-i-get-a-widebar-without-using-the-mathabx-package
\makeatletter
\let\save@mathaccent\mathaccent
\newcommand*\if@single[3]{%
  \setbox0\hbox{${\mathaccent"0362{#1}}^H$}%
  \setbox2\hbox{${\mathaccent"0362{\kern0pt#1}}^H$}%
  \ifdim\ht0=\ht2 #3\else #2\fi
  }
%The bar will be moved to the right by a half of \macc@kerna, which is computed
%by amsmath:
\newcommand*\rel@kern[1]{\kern#1\dimexpr\macc@kerna}
%If there's a superscript following the bar, then no negative kern may follow
%the bar;
%an additional {} makes sure that the superscript is high enough in this case:
\newcommand*\widebar[1]{\@ifnextchar^{{\wide@bar{#1}{0}}}{\wide@bar{#1}{1}}}
%Use a separate algorithm for single symbols:
\newcommand*\wide@bar[2]{\if@single{#1}{\wide@bar@{#1}{#2}{1}}{\wide@bar@{#1}{#2}{2}}}
\newcommand*\wide@bar@[3]{%
  \begingroup
  \def\mathaccent##1##2{%
%Enable nesting of accents:
    \let\mathaccent\save@mathaccent
%If there's more than a single symbol, use the first character instead (see
%below):
    \if#32 \let\macc@nucleus\first@char \fi
%Determine the italic correction:
    \setbox\z@\hbox{$\macc@style{\macc@nucleus}_{}$}%
    \setbox\tw@\hbox{$\macc@style{\macc@nucleus}{}_{}$}%
    \dimen@\wd\tw@
    \advance\dimen@-\wd\z@
%Now \dimen@ is the italic correction of the symbol.
    \divide\dimen@ 3
    \@tempdima\wd\tw@
    \advance\@tempdima-\scriptspace
%Now \@tempdima is the width of the symbol.
    \divide\@tempdima 10
    \advance\dimen@-\@tempdima
%Now \dimen@ = (italic correction / 3) - (Breite / 10)
    \ifdim\dimen@>\z@ \dimen@0pt\fi
%The bar will be shortened in the case \dimen@<0 !
    \rel@kern{0.6}\kern-\dimen@
    \if#31
      \overline{\rel@kern{-0.6}\kern\dimen@\macc@nucleus\rel@kern{0.4}\kern\dimen@}%
      \advance\dimen@0.4\dimexpr\macc@kerna
%Place the combined final kern (-\dimen@) if it is >0 or if a superscript
%follows:
      \let\final@kern#2%
      \ifdim\dimen@<\z@ \let\final@kern1\fi
      \if\final@kern1 \kern-\dimen@\fi
    \else
      \overline{\rel@kern{-0.6}\kern\dimen@#1}%
    \fi
  }%
  \macc@depth\@ne
  \let\math@bgroup\@empty \let\math@egroup\macc@set@skewchar
  \mathsurround\z@ \frozen@everymath{\mathgroup\macc@group\relax}%
  \macc@set@skewchar\relax
  \let\mathaccentV\macc@nested@a
%The following initialises \macc@kerna and calls \mathaccent:
  \if#31
    \macc@nested@a\relax111{#1}%
  \else
%If the argument consists of more than one symbol, and if the first token is
%a letter, use that letter for the computations:
    \def\gobble@till@marker##1\endmarker{}%
    \futurelet\first@char\gobble@till@marker#1\endmarker
    \ifcat\noexpand\first@char A\else
      \def\first@char{}%
    \fi
    \macc@nested@a\relax111{\first@char}%
  \fi
  \endgroup
}
\makeatother

% fat colon
\DeclareFontEncoding{LS1}{}{}
\DeclareFontSubstitution{LS1}{stix}{m}{n}
\DeclareSymbolFont{symbols2}{LS1}{stixfrak}{m}{n}
\DeclareMathSymbol{\typecolon}{\mathbin}{symbols2}{"25}

% ceiling and floor
%\DeclarePairedDelimiter{\ceil}{\lceil}{\rceil}
%\DeclarePairedDelimiter{\floor}{\lfloor}{\rfloor}


%%%%%%%%%%%%%%%%%%%%%%%%%%%%%%%%%%%%%%%%%%%%%%%%%%%%%%%%%%

\newcommand{\slashslash}{//\xspace}
\newcommand{\spplus}{\textsuperscript{+}}
\makeatletter
\newcommand*{\Rnum}[1]{\expandafter\@slowromancap\romannumeral #1@}
\makeatother

% parameters in curly brackets
\newcommand{\param}[1]{\{#1\}}

% mathit
\newcommand{\curry}{\mathit{curry}}
\newcommand{\uncurry}{\mathit{uncurry}}
\newcommand{\Nat}{\mathit{Nat}}
\newcommand{\zeroit}{{\mathit{zero}}}
\newcommand{\succit}{\mathit{succ}}
\newcommand{\plusit}{\mathit{plus}}
\newcommand{\nilit}{\mathit{nil}}
\newcommand{\consit}{\mathit{cons}}
\newcommand{\MInt}{\mathit{MInt}}
\newcommand{\Int}{\mathit{Int}}
\newcommand{\minusit}{\mathit{minus}}
\newcommand{\addinvit}{\mathit{inverse}}
\newcommand{\appendit}{\mathit{append}}
\newcommand{\mplus}{\mathit{mplus}}
\newcommand{\mmult}{\mathit{mmult}}
\newcommand{\List}{\mathit{List}}

% mathsf
\newcommand{\listsf}{\mathsf{list}}
\newcommand{\natsf}{\mathsf{nat}}
\newcommand{\intsf}{\mathsf{int}}
\newcommand{\mintsf}{\mathsf{mint}}

% logic names in mathsf
\newcommand{\ML}{\textnormal{$\mathsf{ML}$}\xspace}
\newcommand{\MLmono}{\textnormal{$\mathsf{ML}^\mathsf{\lowercase{mono}}$}\xspace}
\newcommand{\MLtuple}{\textnormal{$\mathsf{ML}^\mathsf{\lowercase{tuple}}$}\xspace}
\newcommand{\MLapp}{\textnormal{$\mathsf{ML}^\mathsf{\lowercase{app}}$}\xspace}

% general math
\newcommand{\imp}{\to}
\newcommand{\dimp}{\leftrightarrow}
\newcommand{\ldot}{\,\mathord{.}\,}
\newcommand{\pset}[1]{\mathcal{P}(#1)}
\newcommand{\FF}{\mathcal{F}}
\newcommand{\FFF}{\mathbb{F}}
\newcommand{\GGG}{\mathbb{G}}
\newcommand{\cln}{\mathord{:}}
\newcommand{\Cln}{\,\mathord{\typecolon}\,}
\newcommand{\FV}{\mathrm{FV}}
\newcommand{\prule}[1]{(\textsc{#1})}
\newcommand{\pto}{\rightharpoonup}

% matching mu-logic
\newcommand{\MmL}{\mathsf{MmL}}
\newcommand{\EVar}{\textsc{EVar}}
\newcommand{\SVar}{\textsc{SVar}}
\newcommand{\rhobar}{\bar{\rho}}
\newcommand{\Pattern}{\textsc{Pattern}}
\newcommand{\sig}{\mathbb{\Sigma}}
\newcommand{\Var}{\textsc{Var}}
\newcommand{\inh}{\mathit{inh}}
\newcommand{\signat}{\sig^\mathit{nat}}
\newcommand{\Sigmanat}{\Sigma^\mathit{nat}}
\newcommand{\Gammanat}{\Gamma^\mathit{nat}}
\newcommand{\arity}{\mathit{arity}}

% functional matching mu-logic
\newcommand{\MmLi}{\AML}
\newcommand{\dummysort}{\star}
\newcommand{\EVari}{{{\textsc{EVar}}}}
\newcommand{\SVari}{{{\textsc{SVar}}}}
\newcommand{\Patterni}{{{\Pattern}}}
\newcommand{\Sigmai}{{{\Sigma}}}
\newcommand{\app}{\mathit{app}}
\newcommand{\appdot}{\mathbin{\sbullet}}
\newcommand{\vDashi}{\vDash_{\MmLi}{}}

% monosorted
\newcommand{\ccln}{\cln} % x \ccln s === x /\ x \in inh(#s)
\newcommand{\mono}{{\mathsf{mono}}}
\newcommand{\trsort}{\widetilde{\mathbb{M}}}
\newcommand{\trsortpat}{\mathbb{M}}
\newcommand{\trsortinfo}{\trsort_2}
\newcommand{\sigsort}{\sig^\mathsf{mono}}
\newcommand{\Varsort}{\Var^\mathsf{mono}}
\newcommand{\Sigmasort}{\Sigma^\mathsf{mono}}
\newcommand{\EVarsort}{\EVar^\mathsf{mono}}
\newcommand{\SVarsort}{\SVar^\mathsf{mono}}
\newcommand{\ssort}[1]{{\sharp #1}}
\newcommand{\Gammasort}{{\Gamma^\mathsf{mono}}}
\newcommand{\Msort}{{M^\mathsf{mono}}}
\newcommand{\Msortu}{{M^\mathsf{mono}_\dummysort}}
\newcommand{\rhosort}{{\rho^{\mathsf{mono}}}}
\newcommand{\rhobarsort}{\widebar{\rhosort}}

% multi-argument applicative
\newcommand{\multsf}{\mathsf{mult}}
\newcommand{\sigmult}{\sig^\mathsf{mult}}
\newcommand{\Varmult}{\Var^\multsf}
\newcommand{\Sigmamult}{\Sigma^\multsf}
\newcommand{\EVarmult}{\EVar^\multsf}
\newcommand{\SVarmult}{\SVar^\multsf}
\newcommand{\tuple}{\mathit{tuple}}
\newcommand{\Mmult}{M^\multsf}
\newcommand{\Mmultu}{{M^\multsf_\dummysort}}

% applicative matching logic
\newcommand{\AML}{{\textnormal{$\mathsf{ML^\mathsf{app}}$}}\xspace}
\newcommand{\trapp}{\mathbb{A}}
\newcommand{\sigapp}{\sig^\mathsf{app}}
\newcommand{\Varapp}{\Var^\mathsf{app}}
\newcommand{\Sigmaapp}{\Sigma^\mathsf{app}}
\newcommand{\EVarapp}{\EVar^\mathsf{app}}
\newcommand{\SVarapp}{\SVar^\mathsf{app}}
\newcommand{\sapp}[1]{{\sharp #1}}
\newcommand{\Gammaapp}{{F^\mathsf{app}}}
\newcommand{\sortnames}{\mathbb{s}}
\newcommand{\papp}{\mathord{\ }}
\newcommand{\Mstar}{M_\star}
\newcommand{\Mapp}{{M^\mathsf{app}}}
\newcommand{\Mappu}{{M^\mathsf{app}_\dummysort}}
\newcommand{\sharpdummysort}{{\sharp\dummysort}}

% hybrid automata in matching logic
\newcommand{\R}{\mathbb{R}}
\newcommand{\Rn}{\R^{n}}
\newcommand{\nonNegR}{\R_{\geq 0}}
\newcommand{\posR}{\R_{> 0}}
\newcommand{\cfg}{\textit{cfg}}
\newcommand{\Cfg}{\textit{Cfg}}
\newcommand{\StH}{S^t_H}
\newcommand{\init}{\textit{init}}
\newcommand{\inv}{\textit{inv}}
\newcommand{\flow}{\textit{flow}}
\newcommand{\event}{\textit{event}}
\newcommand{\transRel}{\xrightarrow{\text{a}}}
\newcommand{\vec}[1]{\overrightarrow{#1}}


