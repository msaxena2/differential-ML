 \documentclass[]{article}

%% Language and font encodings
\usepackage[english]{babel}
\usepackage[utf8x]{inputenc}
\usepackage[T1]{fontenc}
\usepackage{dirtytalk}
\usepackage{mathbbol}

%% Sets page size and margins
\usepackage[a4paper,top=3cm,bottom=2cm,left=3cm,right=3cm,marginparwidth=1.75cm]{geometry}

%% Useful packages
\usepackage{amsmath}
\usepackage{amssymb}
\usepackage{graphicx}
\usepackage[colorinlistoftodos]{todonotes}
\usepackage[colorlinks=true, allcolors=blue]{hyperref}


% scalable bullet (need graphicx package)
\newcommand\sbullet[1][.5]{\mathbin{\vcenter{\hbox{\scalebox{#1}{$\bullet$}}}}}

% widebar (need amsmath package)
% https://tex.stackexchange.com/questions/16337/
%   can-i-get-a-widebar-without-using-the-mathabx-package
\makeatletter
\let\save@mathaccent\mathaccent
\newcommand*\if@single[3]{%
  \setbox0\hbox{${\mathaccent"0362{#1}}^H$}%
  \setbox2\hbox{${\mathaccent"0362{\kern0pt#1}}^H$}%
  \ifdim\ht0=\ht2 #3\else #2\fi
  }
%The bar will be moved to the right by a half of \macc@kerna, which is computed
%by amsmath:
\newcommand*\rel@kern[1]{\kern#1\dimexpr\macc@kerna}
%If there's a superscript following the bar, then no negative kern may follow
%the bar;
%an additional {} makes sure that the superscript is high enough in this case:
\newcommand*\widebar[1]{\@ifnextchar^{{\wide@bar{#1}{0}}}{\wide@bar{#1}{1}}}
%Use a separate algorithm for single symbols:
\newcommand*\wide@bar[2]{\if@single{#1}{\wide@bar@{#1}{#2}{1}}{\wide@bar@{#1}{#2}{2}}}
\newcommand*\wide@bar@[3]{%
  \begingroup
  \def\mathaccent##1##2{%
%Enable nesting of accents:
    \let\mathaccent\save@mathaccent
%If there's more than a single symbol, use the first character instead (see
%below):
    \if#32 \let\macc@nucleus\first@char \fi
%Determine the italic correction:
    \setbox\z@\hbox{$\macc@style{\macc@nucleus}_{}$}%
    \setbox\tw@\hbox{$\macc@style{\macc@nucleus}{}_{}$}%
    \dimen@\wd\tw@
    \advance\dimen@-\wd\z@
%Now \dimen@ is the italic correction of the symbol.
    \divide\dimen@ 3
    \@tempdima\wd\tw@
    \advance\@tempdima-\scriptspace
%Now \@tempdima is the width of the symbol.
    \divide\@tempdima 10
    \advance\dimen@-\@tempdima
%Now \dimen@ = (italic correction / 3) - (Breite / 10)
    \ifdim\dimen@>\z@ \dimen@0pt\fi
%The bar will be shortened in the case \dimen@<0 !
    \rel@kern{0.6}\kern-\dimen@
    \if#31
      \overline{\rel@kern{-0.6}\kern\dimen@\macc@nucleus\rel@kern{0.4}\kern\dimen@}%
      \advance\dimen@0.4\dimexpr\macc@kerna
%Place the combined final kern (-\dimen@) if it is >0 or if a superscript
%follows:
      \let\final@kern#2%
      \ifdim\dimen@<\z@ \let\final@kern1\fi
      \if\final@kern1 \kern-\dimen@\fi
    \else
      \overline{\rel@kern{-0.6}\kern\dimen@#1}%
    \fi
  }%
  \macc@depth\@ne
  \let\math@bgroup\@empty \let\math@egroup\macc@set@skewchar
  \mathsurround\z@ \frozen@everymath{\mathgroup\macc@group\relax}%
  \macc@set@skewchar\relax
  \let\mathaccentV\macc@nested@a
%The following initialises \macc@kerna and calls \mathaccent:
  \if#31
    \macc@nested@a\relax111{#1}%
  \else
%If the argument consists of more than one symbol, and if the first token is
%a letter, use that letter for the computations:
    \def\gobble@till@marker##1\endmarker{}%
    \futurelet\first@char\gobble@till@marker#1\endmarker
    \ifcat\noexpand\first@char A\else
      \def\first@char{}%
    \fi
    \macc@nested@a\relax111{\first@char}%
  \fi
  \endgroup
}
\makeatother

% fat colon
\DeclareFontEncoding{LS1}{}{}
\DeclareFontSubstitution{LS1}{stix}{m}{n}
\DeclareSymbolFont{symbols2}{LS1}{stixfrak}{m}{n}
\DeclareMathSymbol{\typecolon}{\mathbin}{symbols2}{"25}

% ceiling and floor
%\DeclarePairedDelimiter{\ceil}{\lceil}{\rceil}
%\DeclarePairedDelimiter{\floor}{\lfloor}{\rfloor}


%%%%%%%%%%%%%%%%%%%%%%%%%%%%%%%%%%%%%%%%%%%%%%%%%%%%%%%%%%

\newcommand{\slashslash}{//\xspace}
\newcommand{\spplus}{\textsuperscript{+}}
\makeatletter
\newcommand*{\Rnum}[1]{\expandafter\@slowromancap\romannumeral #1@}
\makeatother

% parameters in curly brackets
\newcommand{\param}[1]{\{#1\}}

% mathit
\newcommand{\curry}{\mathit{curry}}
\newcommand{\uncurry}{\mathit{uncurry}}
\newcommand{\Nat}{\mathit{Nat}}
\newcommand{\zeroit}{{\mathit{zero}}}
\newcommand{\succit}{\mathit{succ}}
\newcommand{\plusit}{\mathit{plus}}
\newcommand{\nilit}{\mathit{nil}}
\newcommand{\consit}{\mathit{cons}}
\newcommand{\MInt}{\mathit{MInt}}
\newcommand{\Int}{\mathit{Int}}
\newcommand{\minusit}{\mathit{minus}}
\newcommand{\addinvit}{\mathit{inverse}}
\newcommand{\appendit}{\mathit{append}}
\newcommand{\mplus}{\mathit{mplus}}
\newcommand{\mmult}{\mathit{mmult}}
\newcommand{\List}{\mathit{List}}

% mathsf
\newcommand{\listsf}{\mathsf{list}}
\newcommand{\natsf}{\mathsf{nat}}
\newcommand{\intsf}{\mathsf{int}}
\newcommand{\mintsf}{\mathsf{mint}}

% logic names in mathsf
\newcommand{\ML}{\textnormal{$\mathsf{ML}$}\xspace}
\newcommand{\MLmono}{\textnormal{$\mathsf{ML}^\mathsf{\lowercase{mono}}$}\xspace}
\newcommand{\MLtuple}{\textnormal{$\mathsf{ML}^\mathsf{\lowercase{tuple}}$}\xspace}
\newcommand{\MLapp}{\textnormal{$\mathsf{ML}^\mathsf{\lowercase{app}}$}\xspace}

% general math
\newcommand{\imp}{\to}
\newcommand{\dimp}{\leftrightarrow}
\newcommand{\ldot}{\,\mathord{.}\,}
\newcommand{\pset}[1]{\mathcal{P}(#1)}
\newcommand{\FF}{\mathcal{F}}
\newcommand{\FFF}{\mathbb{F}}
\newcommand{\GGG}{\mathbb{G}}
\newcommand{\cln}{\mathord{:}}
\newcommand{\Cln}{\,\mathord{\typecolon}\,}
\newcommand{\FV}{\mathrm{FV}}
\newcommand{\prule}[1]{(\textsc{#1})}
\newcommand{\pto}{\rightharpoonup}

% matching mu-logic
\newcommand{\MmL}{\mathsf{MmL}}
\newcommand{\EVar}{\textsc{EVar}}
\newcommand{\SVar}{\textsc{SVar}}
\newcommand{\rhobar}{\bar{\rho}}
\newcommand{\Pattern}{\textsc{Pattern}}
\newcommand{\sig}{\mathbb{\Sigma}}
\newcommand{\Var}{\textsc{Var}}
\newcommand{\inh}{\mathit{inh}}
\newcommand{\signat}{\sig^\mathit{nat}}
\newcommand{\Sigmanat}{\Sigma^\mathit{nat}}
\newcommand{\Gammanat}{\Gamma^\mathit{nat}}
\newcommand{\arity}{\mathit{arity}}

% functional matching mu-logic
\newcommand{\MmLi}{\AML}
\newcommand{\dummysort}{\star}
\newcommand{\EVari}{{{\textsc{EVar}}}}
\newcommand{\SVari}{{{\textsc{SVar}}}}
\newcommand{\Patterni}{{{\Pattern}}}
\newcommand{\Sigmai}{{{\Sigma}}}
\newcommand{\app}{\mathit{app}}
\newcommand{\appdot}{\mathbin{\sbullet}}
\newcommand{\vDashi}{\vDash_{\MmLi}{}}

% monosorted
\newcommand{\ccln}{\cln} % x \ccln s === x /\ x \in inh(#s)
\newcommand{\mono}{{\mathsf{mono}}}
\newcommand{\trsort}{\widetilde{\mathbb{M}}}
\newcommand{\trsortpat}{\mathbb{M}}
\newcommand{\trsortinfo}{\trsort_2}
\newcommand{\sigsort}{\sig^\mathsf{mono}}
\newcommand{\Varsort}{\Var^\mathsf{mono}}
\newcommand{\Sigmasort}{\Sigma^\mathsf{mono}}
\newcommand{\EVarsort}{\EVar^\mathsf{mono}}
\newcommand{\SVarsort}{\SVar^\mathsf{mono}}
\newcommand{\ssort}[1]{{\sharp #1}}
\newcommand{\Gammasort}{{\Gamma^\mathsf{mono}}}
\newcommand{\Msort}{{M^\mathsf{mono}}}
\newcommand{\Msortu}{{M^\mathsf{mono}_\dummysort}}
\newcommand{\rhosort}{{\rho^{\mathsf{mono}}}}
\newcommand{\rhobarsort}{\widebar{\rhosort}}

% multi-argument applicative
\newcommand{\multsf}{\mathsf{mult}}
\newcommand{\sigmult}{\sig^\mathsf{mult}}
\newcommand{\Varmult}{\Var^\multsf}
\newcommand{\Sigmamult}{\Sigma^\multsf}
\newcommand{\EVarmult}{\EVar^\multsf}
\newcommand{\SVarmult}{\SVar^\multsf}
\newcommand{\tuple}{\mathit{tuple}}
\newcommand{\Mmult}{M^\multsf}
\newcommand{\Mmultu}{{M^\multsf_\dummysort}}

% applicative matching logic
\newcommand{\AML}{{\textnormal{$\mathsf{ML^\mathsf{app}}$}}\xspace}
\newcommand{\trapp}{\mathbb{A}}
\newcommand{\sigapp}{\sig^\mathsf{app}}
\newcommand{\Varapp}{\Var^\mathsf{app}}
\newcommand{\Sigmaapp}{\Sigma^\mathsf{app}}
\newcommand{\EVarapp}{\EVar^\mathsf{app}}
\newcommand{\SVarapp}{\SVar^\mathsf{app}}
\newcommand{\sapp}[1]{{\sharp #1}}
\newcommand{\Gammaapp}{{F^\mathsf{app}}}
\newcommand{\sortnames}{\mathbb{s}}
\newcommand{\papp}{\mathord{\ }}
\newcommand{\Mstar}{M_\star}
\newcommand{\Mapp}{{M^\mathsf{app}}}
\newcommand{\Mappu}{{M^\mathsf{app}_\dummysort}}
\newcommand{\sharpdummysort}{{\sharp\dummysort}}

% hybrid automata in matching logic
\newcommand{\R}{\mathbb{R}}
\newcommand{\Rn}{\R^{n}}
\newcommand{\nonNegR}{\R_{\geq 0}}
\newcommand{\posR}{\R_{> 0}}
\newcommand{\cfg}{\textit{cfg}}
\newcommand{\Cfg}{\textit{Cfg}}
\newcommand{\StH}{S^t_H}
\newcommand{\init}{\textit{init}}
\newcommand{\inv}{\textit{inv}}
\newcommand{\flow}{\textit{flow}}
\newcommand{\event}{\textit{event}}
\newcommand{\transRel}{\xrightarrow{\text{a}}}




\newcommand{\SortR}{S_{\R \rightarrow \R}}
\newcommand{\differential}{\textit{differential}}

\title{Differential Equations using Matching $\mu$-logic}

\begin{document}
\maketitle

\section{Differentiation and Differentiable functions}

Asssume sorts $\R$ and $\R$. Let $\nonNegR$ be the sort of non-negative
reals, and $\posR$ the sort of positive reals.
Given a function symbol $f \in \Sigma^{\nonNegR, \R}$,
we define a symbol $\dot{f} \ \cln \ \posR, \R$ to be the first
order derivative of a function w.r.t time, as follows -

% For a given $f \in \Sigma_{\nonNegR, \Rn}$, we define symbol $\dot{f} \in
% \Sigma_{\posR, \Rn}$ as follows -

$$ \forall t \in \posR \ldot (\forall \epsilon : (\nonNegR) \ldot \exists \delta :
\nonNegR \ldot (\forall 0 < h \leq \delta \ldot \mid \frac{f(t + h) -f(t)}{h} -
\dot{f}(t) \mid ) \; \leq \; \epsilon) $$



% Consider $\SortR$ to be the sort of
% functions from $\R \rightarrow \R$. We define a symbol $\differential : \SortR
% \rightarrow \SortR$ to be the first order derivative of a function w.r.t. to time.
% We use $f'$ as syntactic sugar for $\differential(f)$.
%
% Assume a predicate symbol $\textit{differentiable}$, which for a given function
% $f$ is true iff $f$ is differentiable. We define $\differential$ as follows -
%
% $$ \forall f:\SortR\ \textit{differentiable}(f) \implies
%   \exists y:\R . \forall t \geq 0 (\forall \epsilon : \R . \exists \delta : \R .
%       (\forall 0 < h \leq \delta \;  \frac{f(t + h) -f(t)}{h}) = \mid f'(t) - \epsilon \mid = y) $$
%
%
% \subsection{Differential Equations}
% Given a function $\varphi:\SortR$, we define a symbol
% $\int:\SortR \rightarrow \mathcal{P}(\SortR) $ as $\exists f . (\int(f))' =
% \varphi$ (we naturally extend the $'$ symbol to operate over sets of functions).
% Intuitively, $\int$ is the set of functions obtained by integrating $\varphi$.
%
% We also define a symbol $solution: \SortR \times \R \rightarrow \SortR$ as
% follows:
% $$ \forall c \geq 0 \exists f:\SortR . solution(\varphi, c) = f \wedge f' =
% \varphi$$.
%

\section{Preliminaries}
\subsection{Hybrid Automata}
Hybrid Automaton $H$ consists of -
\begin{itemize}
  \item \textbf{(Variables)} A finite set $X = \{x_1, x_2, \dots, x_n\} $ of real valued variables,
    where $n$ is defined as the $dimensionality$ of $H$.
     $\dot{X} = \{\dot{x_1}, \dot{x_2}, \dots, \dot{x_n} \}$ represents the
     the first derivatives during continuous change, and $X' = \{x_1', x_2',
     \dots, x_n' \}$ represents values at end of discrete change.
   \item \textbf{(Control Graph)}. A finite directed multigraph $(V,E)$.
     Vertices are referred to as \textit{Control modes} and edges as
     \textit{Control Switches}.
   \item \textbf{(Initial, Invariant, Flow)}. Vertex labeling functions
     \textit{init}, \textit{inv}, and \textit{flow}. For each $v \in V$,
     $\init(v)$ and $\inv(v)$ assign $v$ a predicate with free variables from
     $X$, while $\flow(v)$ assigns $v$ a predicate with free variables from $X
     \cup X'$
   \item \textbf{(Jump)}. Edge labeling function that for $e \in E$, assigns $e$
     a predicate over $X \cup X'$.
   \item \textbf{(Events)}. A finite set $\Sigma$ of events, and an edge
     labeling function $\event \cln E \to \Sigma$, that assigns to each \textit{control
     switch} $e \in E$, an event $s \in \Sigma$. Note that both $\Sigma$ and $E$
     are finite, and an edge $e \in E$ is also called a \textit{control switch}.

\end{itemize}

\subsection{Labeled transition systems}
For the purpose of this discussion, a labeled transition system $S$ consists of the following -
\begin{itemize}
  \item A possibly infinite set of states $Q$, and set $Q_0 \subseteq Q$ of
    initial states.
  \item A possibly infinite set $A$ of labels, and for each label $a \in A$, a
    binary relation $\transRel$ on $Q$.
\end{itemize}


\subsection{Transition Semantics of Hybrid Automata}
Define a \textit{timed transition system} $\StH$ of a given hybrid automaton
$H$ to capture transition semantics of $H$.
\begin{itemize}
  \item Define $Q, Q_0 \subseteq (V \times \Rn)$. Say, $(v, \overarr{x}) \in Q$ iff
    $\inv(v)[\overarr{x} / X]$ and $(v, \overarr{x}) \in Q_0$ iff $\init(v)[\overarr{x} /
    X]$.
  \item Define $A = \Sigma \cup \nonNegR $. Note $\nonNegR$ denotes the \say{timed}
    part of the $H$.
  \item For given $\sigma \in \Sigma$, say $(v,x) \transRelSigma (v', x')$ iff
    the following hold -
    \begin{itemize}
      \item There exists \textit{control switch} $e$ source $v$ and destination
        $v'$.
      \item $\textit{jump}(e)[\overarr{x} / X]$ holds, and $\textit{event}(e) = \sigma$.
    \end{itemize}
  \item For each non negative real $\delta \in \nonNegR$, say $(v,\overarr{x})
    \transRelSigma (v', \overarr{x'})$ iff
    \begin{itemize}
      \item $ v = v' $
      \item There exists a differentiable function $f \cln [0, \delta] \to \Rn$
        where $\dot{f} \cln (0, \delta) \to \Rn$ is the first order
        derivative of $f$. $f(0) = \overarr{x}$ and $f(\delta) = \overarr{x'}$
        hold.
      \item $\forall \epsilon \in (0, \delta)$, $\inv(v)[f(\epsilon) / X]$
        and $\flow(v)[f(\epsilon) / X, \dot{f}(\epsilon) /  \dot{X}]$ hold.
    \end{itemize}
\end{itemize}



\subsection{Embedding in ML}
Our goal is, given a hybrid automaton $H$ with dimensionality $n$, to define a Mathcing $\mu$-logic
theory $(\Sigma, A)$, s.t. any model $M \vDash (\Sigma, A)$ captures precisely
the timed transition system semantics $\StH$ of $H$.

\begin{itemize}
  \item Define $\boldH{\Sigma} = (\{V, E, \Rn, \R, \sortA\}, \VarH, \Sigma_H)$, where
    \begin{itemize}
      \item $V, E$ are the sorts of $H$'s finite directed graph $(V,E)$.
      \item $\sortEvents$ the finite sort of $H$-events, and $\sortA =
        \sortEvents \cup \nonNegR$.
      \item A sort-indexed countably infinite set of variables $\VarH$ where
        $X\cln \Rn, \dot{X} \cln \Rn X' \cln \Rn \in \VarH_{\Rn} $.
      \item $\Sigma_H$ extends $\Sigma_{\Rn}$, where
        $\Sigma_{\Rn}$ is assumed to be a signature over $\Rn$, an
        n-dimensional real space.
    \end{itemize}
  \item Define $\sortCfg \equiv (V \times \Rn)$. Say $\sbullet \in
    \Sigma_{\sortA \sortCfg, \sortCfg}$
  \item For each $v \in V$, define -
    \begin{itemize}
      \item $\phiInv(v) \cln \Rn \equiv \inv(v)$ where $FV(\phiInv(v)) \subseteq
        X$.
      \item $\phiInit(v) \cln \Rn \equiv \init(v)$ where $FV(\phiInit(v))
        \subseteq X$.
      \item $\phiFlow(v) \cln \Rn \equiv \flow(v)$, where $FV(\phiFlow(v)) \subseteq X \cup \dot{X}$.
    \end{itemize}
  \item For each $e \in E$, $\phiJump(v,v',e) \cln (V \times V \times \Rn) \equiv
        (\textit{source}(e), \textit{destination}(e), \textit{jump}(e)) $ where
        $FV(\phiJump(v,v',e)) \subseteq X \cup X'$ (we
        specifically mean the free variables of the jump condition).
\end{itemize}

Add the following axiomatization to capture $\StH$.
\begin{itemize}
  \item (\textbf{Jump}) For each jump symbol $\phiJump(v,v',e)$
    $$ ((\phiJump(v,v',e))_{\pi_1} \overarr{x}) \to
    \sbullet(event(\phiJump(v,v',e)_{\pi_3}), (\phiJump(v,v',e)_{\pi_2},
    \phiJump(v,v',e)_{\pi_3})) $$.
  \item (\textbf{Flow})

\end{itemize}



% Assume real numbers, or sorts $\R$, $\Rn$. We define a sort $\Cfg \equiv (V
% \times \Rn)$. Define symbol $\sbullet \in \Sigma_{A \Cfg, \Cfg}$, where $A
% \equiv \Sigma \cup \R$.
%
% Use the $\sbullet$ to capture the transition semantics of HAs as follows -
% \begin{itemize}
%   \item
%   \item For every $a \in A$, defined in the given automaton $HA$ is a binary
%     relation on $Q$.
% \end{itemize}


\end{document}
